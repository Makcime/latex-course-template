
\section{Nouveaux environnements}

\begin{defn}[Ensemble Convexe]{def:convex}
	Un ensemble \( S \subseteq \mathbb{R}^n \) est convexe si pour tout \( x, y \in S \) et \( \lambda \in [0, 1] \), le point \( \lambda x + (1-\lambda)y \) appartient à \( S \).
\end{defn}

\begin{prop}[Somme des Angles]{prop:angles}
	La somme des angles dans un triangle est \(180^\circ\).
\end{prop}

\begin{meth}[Méthode de Résolution]{meth:method}
	Pour résoudre une équation linéaire, suivez ces étapes...
\end{meth}

\begin{demo}[Preuve de la Somme des Angles]{demo:proofangles}
	Considérons un triangle quelconque...
\end{demo}

\begin{rem}[Importance de la Convexité]{rem:convexity}
	La convexité est une propriété clé en optimisation.
\end{rem}

\begin{exemple}[Un Ensemble non Convexe]{ex:nonconvexe}
	L'ensemble \( S = \{ (x,y) \mid x^2 + y^2 \leq 1 \} \cup \{(x,y) \mid x^2 + y^2 \geq 4 \} \) n'est pas convexe.
\end{exemple}
